\documentclass[letterpaper]{article} 
\usepackage[utf8]{inputenc}
\linespread{0.85}
\usepackage[T1]{fontenc}
\usepackage{amsmath}
\usepackage{amsfonts}
\usepackage{amssymb}
\usepackage{array}
\usepackage{booktabs}
\usepackage{hyperref}
\usepackage[version=4]{mhchem}
\usepackage{stmaryrd}
\usepackage{tikz}
\usepackage{amsmath}
\usepackage{graphicx}
\usepackage{capt-of}
\usepackage{lipsum}
\usepackage{fancyvrb}
\usepackage{tabularx}
\usepackage{listings}
\usepackage[export]{adjustbox}
\graphicspath{ {./images/} }
\usepackage[utf8]{inputenc}
\usepackage[english]{babel}
\usepackage{float}
\usepackage{lipsum}
\usepackage{graphicx}
\usepackage{float}
\usepackage[margin=0.7in]{geometry}
\usepackage{amsmath}
\usepackage{graphicx}
\usepackage{capt-of}
\usepackage{tcolorbox}
\usepackage{lipsum}
\usepackage{graphicx}
\usepackage{float}
\usepackage{listings}
\usepackage{hyperref} 
\usepackage{xcolor} % For custom colors
\lstset{
	language=Python,                % Choose the language (e.g., Python, C, R)
	basicstyle=\ttfamily\small, % Font size and type
	keywordstyle=\color{blue},  % Keywords color
	commentstyle=\color{gray},  % Comments color
	stringstyle=\color{red},    % String color
	numbers=left,               % Line numbers
	numberstyle=\tiny\color{gray}, % Line number style
	stepnumber=1,               % Numbering step
	breaklines=true,            % Auto line break
	backgroundcolor=\color{black!5}, % Light gray background
	frame=single,               % Frame around the code
}
\usepackage{float}
\usepackage[]{amsthm} %lets us use \begin{proof}
	\usepackage[]{amssymb} %gives us the character \varnothing
	
	\title{Homework 1, IEOR 6614}
	\author{Zongyi Liu}
	\date{Wed, Jan 28, 2026}
	\begin{document}
		\maketitle
		
		\section{Question 1}
		A spanning tree $T$ is a \emph{bottleneck spanning tree} if the maximum arc cost in
		$T$ is as small as possible from among all spanning trees.
		Show that a minimum spanning tree of $G$ is also a bottleneck spanning tree of $G$.
		Is the converse also true? Why or why not?
		
		\textbf{Answer}
		
		
		Let $T$ be an MST of $G$ and let $
		b = \max_{e\in T} w(e).$ Choose an edge $e^\ast \in T$ with $w(e^\ast)=b$.
		Removing $e^\ast$ partitions $T$ into two components with vertex sets
		$(S, V\setminus S)$, defining a cut of $G$.
		By the cut property, $e^\ast$ is a minimum-weight edge crossing this cut; otherwise
		there would exist an edge $f$ crossing the same cut with $w(f) < w(e^\ast)$, and
		$T - e^\ast + f$ would be a spanning tree of smaller total weight, contradicting
		the minimality of $T$.
		Hence every edge crossing the cut has weight at least $b$.
		
		Any spanning tree must contain some edge crossing this cut, and therefore must
		contain an edge of weight at least $b$.
		Thus no spanning tree can have maximum edge weight smaller than $b$, and $T$
		minimizes the maximum edge weight among all spanning trees.
		\hfill$\square$
		
		\medskip
		
		\emph{Converse:} 
		The converse is not true.
		A bottleneck spanning tree minimizes only the maximum edge weight and does not
		necessarily minimize the total weight.
		There may exist spanning trees with the same minimum bottleneck value but with
		larger total weight than an MST; such trees are bottleneck spanning trees but are
		not minimum spanning trees.
		\clearpage
		
	
	\section{Question 2}
	
	Suppose, in the linear-time minimum spanning tree algorithm, we don’t start with
	running 3 iterations of the Bor{\r u}vka algorithm, but instead start with 1.
	Is the resulting algorithm still linear time?
	Either prove that it is, or explain how the analysis from class breaks down.
	What if we run 2 iterations of the Boruvka algorithm?
	How about 4?
	
		\textbf{Answer}

		In the Karger--Klein--Tarjan style linear-time MST algorithm, we use a constant
		number $k$ of Bor{\r u}vka iterations to shrink the number of vertices/components,
		then apply the sampling--filtering step.
		
		Each Bor{\r u}vka iteration can be implemented in $O(m)$ time and reduces the number
		of components (hence contracted vertices) by at least a factor of $2$.
		Thus after $k$ iterations, the contracted graph has $
		n' \;\le\; \frac{n}{2^k}$ vertices.
		
		The sampling lemma used in class implies that after sampling and filtering with
		respect to the MST of the sampled graph, the number of remaining (``$F$-light'')
		edges is $O(n')$ in expectation.
		Hence (writing $T(m,n)$ for expected running time) we obtain a recurrence of the form $
		T(m,n)\;\le\; T\!\left(\frac{m}{2},\,\frac{n}{2^k}\right)\;+\;T\!\left(c\frac{n}{2^k},\,\frac{n}{2^k}\right)\;+\;O(m),$ for some constant $c$, where the $O(m)$ term accounts for the Bor{\r u}vka work,
		sampling, filtering, and contractions at the current level.
	
		Unrolling the recurrence over the ``$m/2$'' branch gives an $O(m)$ geometric series:
		\[
		O(m) + O\!\left(\frac{m}{2}\right) + O\!\left(\frac{m}{4}\right)+\cdots = O(m).
		\]
		For the ``$O(n')$-edge'' branch, the vertex parameter shrinks by a factor $2^k$ each
		level along the recursion, so the total contribution is
		\[
		O\!\left(\frac{n}{2^k}\right)\;+\;O\!\left(\frac{n}{2^{2k}}\right)\;+\;O\!\left(\frac{n}{2^{3k}}\right)\;+\cdots
		\;=\;O(n)
		\quad\text{for any constant }k\ge 1.
		\]
		Therefore, for any constant $k\ge 1$, $
		T(m,n)=O(m+n),$ i.e.\ the algorithm remains linear time (in expectation) even if we start with only
		$k=1$ Bor{\r u}vka iteration.
		
		We consider several cases:
		\begin{itemize}
			\item \emph{1 iteration ($k=1$):} still linear time. The analysis works because the
			$n$-terms form the geometric series $n/2+n/4+n/8+\cdots=O(n)$.
			
			\item \emph{2 iterations ($k=2$):} still linear time, with an even smaller $n$-series
			$n/4+n/16+n/64+\cdots=O(n)$.
			
			\item \emph{4 iterations ($k=4$):} still linear time; this only improves constants,
			since $n' \le n/16$ after the contraction.
		\end{itemize}
		
		If we ran \emph{zero} Bor{\r u}vka iterations ($k=0$), the recurrence would have the form
		\[
		T(m,n)\;\le\;T\!\left(\frac{m}{2},n\right)+T(O(n),n)+O(m),
		\]
		and the $T(O(n),n)$ term would contribute $O(n)$ at each of $\Theta(\log m)$ levels,
		yielding $O(n\log m)$ when $m=O(n)$, i.e.\ not linear.
		This is precisely why we need at least one Bor{\r u}vka iteration, while any
		constant number $k\ge 1$ preserves linear time.
		
		
		\clearpage
		
		\section{Question 3}
		
		The Bellman--Ford algorithm does not specify the order in which to relax edges in
		each pass. Consider the following method for deciding upon the order. Before the
		first pass, assign an arbitrary linear order $v_1, v_2, \ldots, v_{|V|}$ to the
		vertices of the input graph $G = (V,E)$. Then partition the edge set $E$ into
		$E_f \cup E_b$, where
		\[
		E_f = \{(v_i, v_j) \in E \mid i < j\}
		\quad\text{and}\quad
		E_b = \{(v_i, v_j) \in E \mid i > j\}.
		\]
		(Assume that $G$ contains no self-loops, so that every edge belongs to either
		$E_f$ or $E_b$.) Define $G_f = (V, E_f)$ and $G_b = (V, E_b)$.
		
		\begin{enumerate}
			\item Prove that $G_f$ is acyclic with topological sort
			$\langle v_1, v_2, \ldots, v_{|V|} \rangle$ and that $G_b$ is acyclic with
			topological sort $\langle v_{|V|}, v_{|V|-1}, \ldots, v_1 \rangle$.
			
			\item Suppose that each pass of the Bellman--Ford algorithm relaxes edges in the
			following way. First, visit each vertex in the order
			$v_1, v_2, \ldots, v_{|V|}$, relaxing edges of $E_f$ that leave the vertex. Then
			visit each vertex in the order
			$v_{|V|}, v_{|V|-1}, \ldots, v_1$, relaxing edges of $E_b$ that leave the vertex.
			Prove that with this scheme, if $G$ contains no negative-weight cycles that are
			reachable from the source vertex $s$, then after only $\lceil |V|/2 \rceil$
			passes over the edges, $v.d = \delta(s,v)$ for all vertices $v \in V$.
			
			\item Does this scheme improve the asymptotic running time of the Bellman--Ford
			algorithm?
		\end{enumerate}
		
		
		\textbf{Answer}
		
		\subsection{Part 1}
		By definition, every edge in $E_f$ has the form $(v_i,v_j)$ with $i<j$. Hence along
		any directed walk in $G_f$, the vertex indices strictly increase. In particular,
		a directed cycle would force indices to strictly increase and return to the
		starting vertex, which is impossible. Therefore $G_f$ is acyclic. Moreover, the
		order $\langle v_1,v_2,\dots,v_{|V|}\rangle$ is a topological ordering of $G_f$
		because every edge goes from a smaller index to a larger index.
		
		Similarly, every edge in $E_b$ has the form $(v_i,v_j)$ with $i>j$, so along any
		directed walk in $G_b$ the indices strictly decrease. Thus $G_b$ is acyclic, and
		$\langle v_{|V|},v_{|V|-1},\dots,v_1\rangle$ is a topological ordering of $G_b$.
		
		\subsection{Part 2}
		Assume there is no negative-weight cycle reachable from $s$.
		Let a \emph{pass} mean: relax all edges of $E_f$ in the forward vertex order, then
		relax all edges of $E_b$ in the backward vertex order.
		
		\medskip
		
		\emph{Lemma} After the forward sweep of any pass, for every vertex $v_j$,
		the value $d[v_j]$ equals the minimum weight among all $s\!\to\! v_j$ paths whose
		last edge lies in $E_f$; after the backward sweep of that same pass, $d[v_j]$
		equals the minimum weight among all $s\!\to\! v_j$ paths whose last edge lies in
		$E_b$ as well. Consequently, after a full pass,
		\[
		d[v] \;=\; \min\{\text{weight}(P): P \text{ is an } s\!\to\! v \text{ path that can be
			written as } P=P_1P_2\},
		\]
		where $P_1$ uses only edges of $E_f$ and $P_2$ uses only edges of $E_b$ (either part
		may be empty).
		
		\medskip
		
		\emph{Proof of lemma}
		Consider the forward sweep. Since $G_f$ is acyclic with topological order
		$\langle v_1,\dots,v_{|V|}\rangle$, relaxing all outgoing $E_f$-edges in this order
		computes correct shortest-path distances in $G_f$ (standard DP on a DAG) using the
		current $d[\cdot]$ values as sources. Thus, after the forward sweep, $d[v_j]$ is
		the minimum over all paths that end with an $E_f$-edge (or use only $E_f$ after
		leaving $s$). The same argument applies to the backward sweep on the DAG $G_b$
		with topological order $\langle v_{|V|},\dots,v_1\rangle$. \hfill$\square$
		
		\medskip
		
		Now take any vertex $v$ and fix a shortest $s\!\to\! v$ path $P$ in $G$.
		Because there are no reachable negative cycles, $P$ can be chosen \emph{simple},
		hence uses at most $|V|-1$ edges.
		Write the vertices of $P$ in the global order indices; along $P$, every edge is
		either forward (index increases) or backward (index decreases). Let $k(P)$ be the
		number of \emph{direction changes} along $P$ (from forward to backward or vice
		versa). Each direction change requires at least one edge, so
		\[
		k(P) \le |E(P)|-1 \le (|V|-1)-1 = |V|-2.
		\]
		Therefore $P$ has at most $k(P)+1 \le |V|-1$ monotone segments, where each segment
		consists solely of forward edges or solely of backward edges.
		
		A single full pass (forward sweep then backward sweep) can correctly propagate
		shortest-path information across \emph{two} consecutive monotone segments of $P$:
		one forward segment is handled by the forward sweep on $G_f$, and the subsequent
		backward segment is handled by the backward sweep on $G_b$ (the lemma formalizes
		this). Thus after $t$ passes, all shortest paths whose decomposition into monotone
		segments has at most $2t$ segments have been fully propagated, so $d[v]=\delta(s,v)$
		for all vertices reachable by such paths.
		
		Since any simple path has at most $|V|-1$ segments, choosing
		\[
		2t \ge |V|-1 \quad\Longleftrightarrow\quad t \ge \left\lceil \frac{|V|-1}{2}\right\rceil
		\]
		is sufficient. Hence after $\left\lceil |V|/2\right\rceil$ passes, we obtain
		$d[v]=\delta(s,v)$ for all $v\in V$.
		
		\subsection{Part 3}
		Each pass relaxes every edge once, so it costs $\Theta(m)$ time.
		The number of passes is $\left\lceil |V|/2\right\rceil = \Theta(|V|)$, so the total
		running time is $\Theta(m|V|)$, which is the same asymptotic bound as standard
		Bellman--Ford (improving only the constant factor by about $2$).
		
		
		\clearpage
		
		\section{Question 4}
		
		Suppose that you have a graph where the weights are integers between $0$ and $C$
		for some constant $C$. Show how to implement Dijkstra's algorithm in
		$O(n+m)$ time.
		
		\textbf{Answer}
		
		Since all edge weights are integers in $\{0,1,\dots,C\}$ with $C=O(1)$, we can
		implement Dijkstra using a bucketed priority queue (Dial's algorithm).
		
		Maintain an array of buckets $B[0],B[1],\dots,B[Cn]$, where bucket $B[d]$ stores
		all vertices whose current tentative distance is $d$.
		Initialize $d[s]=0$, $d[v]=\infty$ for $v\neq s$, and insert $s$ into $B[0]$.
		Keep a pointer $p$ to the smallest index such that $B[p]$ is nonempty.
		
		Repeatedly extract a vertex $u$ from $B[p]$ (advance $p$ forward until a nonempty
		bucket is found). For each outgoing edge $(u,v)$ with weight $w(u,v)\in[0,C]$,
		perform the relaxation
		\[
		\text{if } d[v] > d[u] + w(u,v)\text{ then set } d[v]\leftarrow d[u]+w(u,v)
		\text{ and move $v$ to bucket } B[d[v]].
		\]
		Each vertex moves buckets only when its distance decreases, and each edge is
		relaxed once when its tail is extracted. The pointer $p$ increases from $0$ to at
		most $Cn$, so the total time spent scanning buckets is $O(Cn)=O(n)$, and the total
		time spent on relaxations is $O(m)$.
		
		Therefore the overall running time is $O(n+m)$.
		
		
		\clearpage
		
		\section{Question 5}
		Give a graph $G$ with non-negative edge weights, give an algorithm to find the
		second shortest path in the graph. Make the algorithm as efficient as you can.
		
		\textbf{Answer}
		
		\emph{Graph}
		Let $V=\{s,a,b,t\}$ and let the (directed or undirected) edges with nonnegative
		weights be
		\[
		(s,a):1,\quad (a,t):1,\quad (s,b):1,\quad (b,t):2,\quad (a,b):0.
		\]
		
		We can plot it as below:
		
		\begin{tikzpicture}[>=stealth, node distance=2.8cm]
			
			% Nodes
			\node[circle, draw, minimum size=8mm] (s) {$s$};
			\node[circle, draw, minimum size=8mm, right of=s, above of=s] (a) {$a$};
			\node[circle, draw, minimum size=8mm, right of=s, below of=s] (b) {$b$};
			\node[circle, draw, minimum size=8mm, right of=a, below of=a] (t) {$t$};
			
			% Edges
			\draw[->] (s) -- node[above left] {$1$} (a);
			\draw[->] (a) -- node[above right] {$1$} (t);
			
			\draw[->] (s) -- node[below left] {$1$} (b);
			\draw[->] (b) -- node[below right] {$2$} (t);
			
			\draw[->] (a) -- node[right] {$0$} (b);
			
		\end{tikzpicture}
		
		Then the shortest $s$--$t$ path is $s\to a\to t$ with length $2$, and the second
		shortest is $s\to a\to b\to t$ with length $3$ (the path $s\to b\to t$ also has
		length $3$).
		
		\bigskip
		
		\emph{Algorithm}
		Maintain two best distances to each vertex:
		\[
		d_1[v]=\text{shortest distance from }s\text{ to }v,\qquad
		d_2[v]=\text{second shortest distance from }s\text{ to }v,
		\]
		initialized by $d_1[s]=0$, $d_2[s]=\infty$, and $d_1[v]=d_2[v]=\infty$ for $v\neq s$.
		Run a Dijkstra-like algorithm with a min-priority queue of pairs $(\ell,v)$:
		
		\begin{enumerate}
			\item Insert $(0,s)$.
			\item While the queue is nonempty, extract-min $(\ell,u)$; if $\ell>d_2[u]$, continue.
			\item For each edge $(u,v)$ with weight $w(u,v)$, set $x=\ell+w(u,v)$ and update:
			\begin{itemize}
				\item If $x<d_1[v]$, set $d_2[v]\leftarrow d_1[v]$, $d_1[v]\leftarrow x$, and push $(d_1[v],v)$.
				\item Else if $d_1[v]<x<d_2[v]$, set $d_2[v]\leftarrow x$ and push $(d_2[v],v)$.
			\end{itemize}
		\end{enumerate}
		
		At the end, $d_1[t]$ is the shortest-path length and $d_2[t]$ is the second
		shortest-path length.
		
		\medskip
		
		\clearpage
		
		
		
		
	\end{document}
