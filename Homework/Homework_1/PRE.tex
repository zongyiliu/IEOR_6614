\documentclass[letterpaper]{article} 
\usepackage[utf8]{inputenc}
\linespread{0.85}
\usepackage[T1]{fontenc}
\usepackage{amsmath}
\usepackage{amsfonts}
\usepackage{amssymb}
\usepackage{array}
\usepackage{booktabs}
\usepackage{hyperref}
\usepackage[version=4]{mhchem}
\usepackage{stmaryrd}
\usepackage{tikz}
\usepackage{amsmath}
\usepackage{graphicx}
\usepackage{capt-of}
\usepackage{lipsum}
\usepackage{fancyvrb}
\usepackage{tabularx}
\usepackage{listings}
\usepackage[export]{adjustbox}
\graphicspath{ {./images/} }
\usepackage[utf8]{inputenc}
\usepackage[english]{babel}
\usepackage{float}
\usepackage{lipsum}
\usepackage{graphicx}
\usepackage{float}
\usepackage[margin=0.7in]{geometry}
\usepackage{amsmath}
\usepackage{graphicx}
\usepackage{capt-of}
\usepackage{tcolorbox}
\usepackage{lipsum}
\usepackage{graphicx}
\usepackage{float}
\usepackage{listings}
\usepackage{hyperref} 
\usepackage{xcolor} % For custom colors
\lstset{
	language=Python,                % Choose the language (e.g., Python, C, R)
	basicstyle=\ttfamily\small, % Font size and type
	keywordstyle=\color{blue},  % Keywords color
	commentstyle=\color{gray},  % Comments color
	stringstyle=\color{red},    % String color
	numbers=left,               % Line numbers
	numberstyle=\tiny\color{gray}, % Line number style
	stepnumber=1,               % Numbering step
	breaklines=true,            % Auto line break
	backgroundcolor=\color{black!5}, % Light gray background
	frame=single,               % Frame around the code
}
\usepackage{float}
\usepackage[]{amsthm} %lets us use \begin{proof}
	\usepackage[]{amssymb} %gives us the character \varnothing
	
	\title{Homework 1, IEOR 6614}
	\author{Zongyi Liu}
	\date{Wed, Jan 28, 2026}
	\begin{document}
		\maketitle
		
		\section{Question 1}
		A spanning tree $T$ is a \emph{bottleneck spanning tree} if the maximum arc cost in
		$T$ is as small as possible from among all spanning trees.
		Show that a minimum spanning tree of $G$ is also a bottleneck spanning tree of $G$.
		Is the converse also true? Why or why not?
		
		\textbf{Answer}
		
		
		\clearpage
		
	
	\section{Question 2}
	
	Suppose, in the linear-time minimum spanning tree algorithm, we don’t start with
	running 3 iterations of the Bor{\r u}vka algorithm, but instead start with 1.
	Is the resulting algorithm still linear time?
	Either prove that it is, or explain how the analysis from class breaks down.
	What if we run 2 iterations of the Boruvka algorithm?
	How about 4?
	
		\textbf{Answer}

		\clearpage
		
		\section{Question 3}
		
		The Bellman--Ford algorithm does not specify the order in which to relax edges in
		each pass. Consider the following method for deciding upon the order. Before the
		first pass, assign an arbitrary linear order $v_1, v_2, \ldots, v_{|V|}$ to the
		vertices of the input graph $G = (V,E)$. Then partition the edge set $E$ into
		$E_f \cup E_b$, where
		\[
		E_f = \{(v_i, v_j) \in E \mid i < j\}
		\quad\text{and}\quad
		E_b = \{(v_i, v_j) \in E \mid i > j\}.
		\]
		(Assume that $G$ contains no self-loops, so that every edge belongs to either
		$E_f$ or $E_b$.) Define $G_f = (V, E_f)$ and $G_b = (V, E_b)$.
		
		\begin{enumerate}
			\item Prove that $G_f$ is acyclic with topological sort
			$\langle v_1, v_2, \ldots, v_{|V|} \rangle$ and that $G_b$ is acyclic with
			topological sort $\langle v_{|V|}, v_{|V|-1}, \ldots, v_1 \rangle$.
			
			\item Suppose that each pass of the Bellman--Ford algorithm relaxes edges in the
			following way. First, visit each vertex in the order
			$v_1, v_2, \ldots, v_{|V|}$, relaxing edges of $E_f$ that leave the vertex. Then
			visit each vertex in the order
			$v_{|V|}, v_{|V|-1}, \ldots, v_1$, relaxing edges of $E_b$ that leave the vertex.
			Prove that with this scheme, if $G$ contains no negative-weight cycles that are
			reachable from the source vertex $s$, then after only $\lceil |V|/2 \rceil$
			passes over the edges, $v.d = \delta(s,v)$ for all vertices $v \in V$.
			
			\item Does this scheme improve the asymptotic running time of the Bellman--Ford
			algorithm?
		\end{enumerate}
		
		
		\textbf{Answer}
		
		\clearpage
		\section{Question 4}
		
		Suppose that you have a graph where the weights are integers between $0$ and $C$
		for some constant $C$. Show how to implement Dijkstra's algorithm in
		$O(n+m)$ time.
		
		\textbf{Answer}
		
		
		
	\end{document}
