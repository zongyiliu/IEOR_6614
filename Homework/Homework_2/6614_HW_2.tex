\documentclass[letterpaper]{article} 
\usepackage[utf8]{inputenc}
\linespread{0.85}
\usepackage[T1]{fontenc}
\usepackage{amsmath}
\usepackage{amsfonts}
\usepackage{amssymb}
\usepackage{array}
\usepackage{booktabs}
\usepackage{hyperref}
\usepackage[version=4]{mhchem}
\usepackage{stmaryrd}
\usepackage{tikz}
\usepackage{amsmath}
\usepackage{graphicx}
\usepackage[ruled,vlined]{algorithm2e}
\usepackage{capt-of}
\usepackage{lipsum}
\usepackage{fancyvrb}
\usepackage{tabularx}
\usepackage{listings}
\usepackage[export]{adjustbox}
\graphicspath{ {./images/} }
\usepackage[utf8]{inputenc}
\usepackage[english]{babel}
\usepackage{float}
\usepackage{lipsum}
\usepackage{graphicx}
\usepackage{float}
\usepackage[margin=0.7in]{geometry}
\usepackage{amsmath}
\usepackage{graphicx}
\usepackage{capt-of}
\usepackage{tcolorbox}
\usepackage{lipsum}
\usepackage{graphicx}
\usepackage{float}
\usepackage{listings}
\usepackage{hyperref} 
\usepackage{fontspec}
\definecolor{deepred}{RGB}{139,0,0}
\usepackage{xcolor} % For custom colors
\lstset{
	language=Python,                % Choose the language (e.g., Python, C, R)
	basicstyle=\ttfamily\small, % Font size and type
	keywordstyle=\color{blue},  % Keywords color
	commentstyle=\color{gray},  % Comments color
	stringstyle=\color{red},    % String color
	numbers=left,               % Line numbers
	numberstyle=\tiny\color{gray}, % Line number style
	stepnumber=1,               % Numbering step
	breaklines=true,            % Auto line break
	backgroundcolor=\color{black!5}, % Light gray background
	frame=single,               % Frame around the code
}
\usepackage{float}
\usepackage[]{amsthm} %lets us use \begin{proof}
	\usepackage[]{amssymb} %gives us the character \varnothing
	
	\title{Homework 2, IEOR 6614}
	\author{Zongyi Liu}
	\date{Wed, Feb 11, 2026}
	\begin{document}
		\maketitle
		
		\section{Question 1}
		
		{\color{deepred}\fontspec{Georgia} Reduction} 
		
		Suppose you do not want to use the minimum-mean cycle algorithm given in class and in the book, but instead have access to a "Negative Cycle Detection" black box (like Bellman-Ford).
		
		\begin{itemize}
			\item Show that for any real number $\lambda$, the problem of determining if $\mu^{*} \leq \lambda$ can be reduced to the problem of detecting a negative cycle in a modified graph.
			\item Using this reduction, describe an algorithm to find an $\epsilon$-approximate value of $\mu^{*}$ using binary search. What is the range of values for the binary search if all edge weights are integers in $[-W, W]$ ?
		\end{itemize}
		
		\textbf{Answer}
		
		First we set up the algorithm, let $G=(V,E)$ be a directed graph with (original) edge weights $w:E\to\mathbb{R}$.
		For a directed cycle $C$, let
		\[
		W(C)=\sum_{e\in C} w(e),\qquad |C|=\text{number of edges in }C,\qquad
		\mu(C)=\frac{W(C)}{|C|}.
		\]
		Define the minimum mean cycle value
		\[
		\mu^*=\min_{C\text{ cycle}} \mu(C).
		\]
		
		
		\subsection{Part 1}
			
			Fix any real $\lambda$. Construct a modified graph $G_\lambda=(V,E)$ with
			\[
			w_\lambda(e)=w(e)-\lambda \qquad\forall e\in E.
			\]
			For any directed cycle $C$,
			\[
			\sum_{e\in C} w_\lambda(e)
			=\sum_{e\in C} (w(e)-\lambda)
			= W(C) - \lambda |C|
			= |C|\left(\mu(C)-\lambda\right).
			\]
			Therefore,
			\[
			\sum_{e\in C} w_\lambda(e) < 0
			\iff |C|(\mu(C)-\lambda)<0
			\iff \mu(C) < \lambda.
			\]
			Hence $G_\lambda$ contains a negative cycle iff there exists a cycle $C$ with
			$\mu(C)<\lambda$, i.e.
			\[
			G_\lambda \text{ has a negative cycle}
			\iff \mu^* < \lambda.
			\]
			In particular, to decide whether $\mu^*\le \lambda$, one can query the black box
			at values arbitrarily close to $\lambda$:
			\[
			\mu^*\le \lambda \iff \neg\big(\mu^*>\lambda\big)
			\iff \neg\big(\mu^*\ge \lambda+\delta \text{ for some }\delta>0\big),
			\]
			and note that
			\[
			\mu^* < x \iff G_x \text{ has a negative cycle}.
			\]
			So we can implement a monotone predicate for binary search, e.g.
			\[
			P(x)=\mathbf{1}\{G_x \text{ has a negative cycle}\}=\mathbf{1}\{\mu^*<x\}.
			\]
			This predicate is monotone: if $P(x)=1$ and $y>x$, then also $P(y)=1$.
			
			\subsection{Part 2}
			
			Assume we have a negative-cycle detector $\textsc{NegCycle}(G,\text{weights})$
			that returns \texttt{true} iff the given weighted directed graph has a negative cycle.
			
			We binary search on $\lambda$ using the monotone predicate
			\[
			P(\lambda):=\textsc{NegCycle}(G, w_\lambda).
			\]
			Maintain an interval $[L,U]$ such that $\mu^*\in [L,U]$ and $U-L\le \epsilon$ at termination.
			Initialize $L,U$ as a valid lower/upper bound on $\mu^*$ (see below).
			
			\medskip
			\noindent{Binary search algorithm can be drafted as follows:}
			\begin{enumerate}
				\item Set $L\gets \underline{B}$, $U\gets \overline{B}$ where $\underline{B}\le \mu^*\le \overline{B}$.
				\item While $U-L>\epsilon$:
				\begin{enumerate}
					\item $\lambda \gets (L+U)/2$.
					\item If $\textsc{NegCycle}(G, w-\lambda)$ returns \texttt{true} (i.e. $P(\lambda)=1$),
					then $\mu^*<\lambda$, so set $U\gets \lambda$.
					\item Else no negative cycle exists in $G_\lambda$, so $\mu^*\ge \lambda$, set $L\gets \lambda$.
				\end{enumerate}
				\item Output any $\hat\mu\in[L,U]$ (e.g. $\hat\mu=(L+U)/2$). Then $|\hat\mu-\mu^*|\le \epsilon$.
			\end{enumerate}
			
			\medskip
			If $P(\lambda)=1$, then $\mu^*<\lambda$ so the upper endpoint can be decreased.
			If $P(\lambda)=0$, then $\mu^*\ge \lambda$ so the lower endpoint can be increased.
			Thus $\mu^*\in[L,U]$ is preserved and the interval halves each iteration.
			The number of iterations is
			\[
			O\!\left(\log\frac{U-L}{\epsilon}\right).
			\]
			
			\medskip
			\noindent{Binary-search range when $w(e)\in[-W,W]\cap\mathbb{Z}$.}
			For any directed cycle $C$,
			\[
			-W \le w(e)\le W \implies -W|C|\le W(C)\le W|C|
			\implies -W \le \mu(C)=\frac{W(C)}{|C|}\le W.
			\]
			Taking the minimum over cycles yields
			\[
			-W \le \mu^* \le W.
			\]
			So we may initialize
			\[
			L=-W,\qquad U=W,
			\]
			and binary search over $[-W,W]$.
		
		
		\clearpage
	
	\section{Question 2}
	
	{\color{deepred}\fontspec{Georgia} Blocking Flow Detection} 
	
	 Show how to find a blocking flow in an acyclic network in $O(n m)$ time by successively augmenting along a path of non-saturated edges and using depth-first search to find such a path. Show how to obtain a running time of $O(m)$ if all edges that are not incident to $s$ or $t$ have capacity 1 .
	
	
		\textbf{Answer}
		
		Let $N=(V,E)$ be an $s$--$t$ flow network with capacities $c_e$ and a current flow $f$.
		Let $G_f$ be the residual graph and assume $G_f$ is \emph{acyclic}.
		A \emph{blocking flow} is an $s$--$t$ flow in $G_f$ such that every $s$--$t$ path in $G_f$
		contains at least one saturated (residual capacity $0$) edge.
		
		
		The blocking flow in an acyclic residual network in $O(nm)$ is as follows:
		
	\noindent
	\fbox{
		\begin{minipage}{0.97\linewidth}
			\textbf{Algorithm}
			
			Maintain residual capacities $r(e)$ for each residual edge $e$.
			
			Repeat:
			\begin{enumerate}
				\item Run a DFS in $G_f$ restricted to \emph{non-saturated} residual edges ($r(e)>0$),
				starting at $s$, to find an $s$--$t$ path $P$.
				\item If no such path exists, stop; the current flow is blocking.
				\item Otherwise augment by
				\[
				\Delta := \min_{e\in P} r(e)
				\]
				along $P$ (decrease $r(e)$ by $\Delta$ on forward edges of $P$ and increase appropriately
				on reverse residual edges).
			\end{enumerate}
		\end{minipage}
	}
	
		When the algorithm stops, DFS found no $s$--$t$ path consisting solely of non-saturated
		(residual-positive) edges, hence every $s$--$t$ path in the residual graph contains at least
		one saturated edge. This is exactly the definition of a blocking flow.
		
		\underline{Running time $O(nm)$.}
		Each DFS takes $O(m)$ time. We claim there are at most $O(n)$ augmentations.
		
		Because $G_f$ is acyclic, fix any topological order of its vertices.
		In each augmentation along a simple $s$--$t$ path $P$, at least one edge of $P$ becomes
		\emph{newly saturated} (some edge achieves $r(e)=0$), namely any edge attaining the bottleneck
		$\Delta$.
		Consider a single augmentation and let $e=(u,v)$ be a newly saturated edge on $P$.
		In an acyclic residual graph, after we saturate $e$ and update residual capacities, $e$
		cannot appear again on any future $s$--$t$ path found by the DFS without first being
		\emph{unsaturated} via sending flow back through its reverse residual edge.
		But sending flow back would require an $s$--$t$ augmentation path that uses the reverse edge
		$(v,u)$; together with the subpath from $u$ to $v$ in that augmentation, this would create a
		directed cycle in the residual graph, contradicting acyclicity.
		Hence, once a residual edge becomes saturated, it is never used again by any later augmenting
		path.
		
		Therefore each augmentation permanently eliminates (at least) one residual edge from future
		search paths. Along any simple $s$--$t$ path there are at most $n-1$ edges, so after at most
		$n-1$ augmentations the DFS cannot keep finding new $s$--$t$ paths without reusing a saturated
		edge (which cannot happen). Thus the number of augmentations is $O(n)$, and the total time is
		\[
		O(\#\text{DFS} \cdot m)=O(n\cdot m).
		\]
		
		\bigskip
		\underline{$O(m)$ time when all internal edges have capacity 1}
		
		Assume every edge not incident to $s$ or $t$ has capacity $1$. On any augmenting $s$--$t$ path, the bottleneck is always $\Delta=1$ unless the only
		bottlenecks are edges incident to $s$ or $t$.
		In particular, any augmentation that saturates an \emph{internal} edge (not incident to $s$ or $t$)
		saturates it completely and (by the same acyclicity argument) removes it forever from future paths.
		So the number of such augmentations is at most the number of internal edges, i.e.\ $O(m)$.
		To get \emph{total} time $O(m)$, we must also make the \emph{sum of search work} $O(m)$.
		
		{Linear-time implementation via DFS with current-arc pointers.} Store adjacency lists and maintain for each vertex $v$ a pointer $\textit{next}(v)$ to the next
		outgoing residual edge in its adjacency list not yet proved useless.
		
		Run a single DFS-like procedure:
		\begin{enumerate}
			\item Start at $s$ and follow outgoing residual edges with $r(e)>0$, advancing
			$\textit{next}(v)$ when an edge is saturated or leads to a dead end.
			\item Maintain a stack representing the current $s$-to-$x$ path in the DFS tree.
			When you reach $t$, augment by $1$ along the stack path (this saturates at least one internal edge),
			then backtrack to the first vertex on the stack that still has unexplored outgoing residual edges.
			\item If you backtrack from a vertex $v$ and $\textit{next}(v)$ is past the end of its adjacency list,
			then $v$ cannot reach $t$ via non-saturated edges; pop it and continue backtracking.
			Stop when the stack becomes empty (no $s$--$t$ path exists), yielding a blocking flow.
		\end{enumerate}
		
		\underline{Why total time is $O(m)$.}
		Each time the pointer $\textit{next}(v)$ advances past an edge $e$, we never consider $e$ again.
		Thus the total number of edge scans over the entire algorithm is $O(m)$.
		Also, each augmentation saturates at least one internal capacity-$1$ edge, which is then never used
		again, so the total number of augmentations is $O(m)$ but does \emph{not} multiply the edge-scan cost,
		since searches continue from where they left off using the current-arc pointers.
		
		Hence the whole procedure performs $O(m)$ pointer advances and $O(m)$ stack operations, giving total
		running time $O(m)$.
		
		Thus, in an acyclic residual network, repeatedly finding a non-saturated $s$--$t$ path by DFS and augmenting
		yields a blocking flow in $O(nm)$ time.
		If all edges not incident to $s$ or $t$ have capacity $1$, the same idea with current-arc pointers
		and a single global DFS traversal runs in $O(m)$ time.
		

		
		\clearpage
		
		\section{Question 3}
		
		{\color{deepred}\fontspec{Georgia} Hoffman's Circulation Theorem \small{(Exercise 4, Chapter 8 from the book of Korte \& Vygen)}} 
		
		 A circulation is an $s-t$ flow with the extra property that the excess at $s$ and $t$ is 0 . Prove Hoffman's circulation theorem: Given a digraph $D(V, A)$ and lower and upper capacities $l, u: E(G) \rightarrow \mathbb{R}_{+}$with $l(a) \leq u(a)$ for all $a \in A$, there is circulation $f$ with $l(a) \leq f(a) \leq u(a)$ for all $a \in E(G)$ if and only if
		
		$$
		\sum_{a \in \delta^{-}(X)} l(a) \leq \sum_{a \in \delta^{+}(X)} u(a) \quad \text { for all } X \subseteq V(G) .
		$$
		
		Note: Hoffman's circulation theorem in turn quite easily implies the Max-Flow Min-Cut Theorem (Hoffman [1960]).
		
		
		\textbf{Answer}
		
		
		Let $D=(V,A)$ be a directed graph with lower and upper capacities
		$l,u : A \to \mathbb{R}_+$ satisfying $l(a) \le u(a)$ for all $a\in A$.
		There exists a circulation $f$ with
		\[
		l(a) \le f(a) \le u(a) \quad \forall a\in A
		\]
		if and only if for every $X\subseteq V$,
		\[
		\sum_{a\in \delta^-(X)} l(a)
		\;\le\;
		\sum_{a\in \delta^+(X)} u(a).
		\]
		
		\bigskip
		
		\underline{Proof}
		
		\medskip
		\emph{(Necessity)}
		Suppose $f$ is a feasible circulation.
		For any $X\subseteq V$, flow conservation gives $
		\sum_{a\in \delta^+(X)} f(a)
		=
		\sum_{a\in \delta^-(X)} f(a).$ 
		
		Since $l(a)\le f(a)\le u(a)$ for all $a$,
		
		\[
		\sum_{a\in \delta^-(X)} l(a)
		\le
		\sum_{a\in \delta^-(X)} f(a)
		=
		\sum_{a\in \delta^+(X)} f(a)
		\le
		\sum_{a\in \delta^+(X)} u(a).\]
		
		Hence the stated inequality holds for all $X$.
		
		\medskip
		\emph{(Sufficiency)}
		Assume the cut inequalities hold for all $X\subseteq V$.
		We reduce the problem to a max-flow instance.
		
		Define new capacities as $
		c(a) := u(a) - l(a) \ge 0.$ We seek $f$ with $l(a)\le f(a)\le u(a)$ and flow conservation.
		Let $
		g(a) := f(a) - l(a).$ Then $0\le g(a)\le c(a)$.
		Flow conservation at a vertex $v$ becomes
		\[
		\sum_{a\in \delta^-(v)} g(a)
		-
		\sum_{a\in \delta^+(v)} g(a)
		=
		b(v),
		\]
		where
		\[
		b(v)
		:=
		\sum_{a\in \delta^+(v)} l(a)
		-
		\sum_{a\in \delta^-(v)} l(a).
		\]
		Observe that $\sum_{v\in V} b(v)=0$.
		
		We now construct an auxiliary network.
		Add a super-source $s$ and super-sink $t$.
		For each vertex $v$:
		
		- If $b(v)>0$, add arc $(s,v)$ with capacity $b(v)$.
		- If $b(v)<0$, add arc $(v,t)$ with capacity $-b(v)$.
		
		All original arcs retain capacity $c(a)$.
		
		A feasible circulation exists if and only if there is a flow
		from $s$ to $t$ saturating all arcs leaving $s$,
		i.e.
		\[
		\text{max flow} = \sum_{b(v)>0} b(v).
		\]
		
		By the Max-Flow Min-Cut Theorem, this holds if and only if
		every $s$–$t$ cut has capacity at least $\sum_{b(v)>0} b(v)$.
		
		Consider an arbitrary cut $(S,T)$ with $s\in S$, $t\in T$.
		Let $X = S\setminus\{s\}$.
		A direct computation shows that the capacity of the cut equals
		\[
		\sum_{a\in \delta^+(X)} (u(a)-l(a))
		+
		\sum_{a\in \delta^-(X)} l(a).
		\]
		Requiring this to be at least $\sum_{b(v)>0} b(v)$
		is equivalent (after simplification) to
		\[
		\sum_{a\in \delta^-(X)} l(a)
		\le
		\sum_{a\in \delta^+(X)} u(a).
		\]
		
		Thus the cut condition guarantees that every $s$–$t$ cut
		has sufficient capacity, so a saturating flow exists,
		and hence a feasible circulation exists.
		
		
		
		
		\clearpage
		
		\section{Question 4}
		
		{\color{deepred}\fontspec{Georgia} Scaling}
		
		In this exercise, we see another modification of the Ford-Fulkerson algorithm that improves its running time.
		
		Given a network $D(V, A)$ where, without loss of generality, we assume all edge capacities $u$ are integers, a feasible flow $f$, and a parameter $\Delta$, define the $\Delta$-residual network as the residual network of $D(V, A)$ with respect to $f$, containing only the arcs with capacity at least $\Delta$. Let $D_{f}(\Delta)$ denote the $\Delta$-residual network. Let $m$ be the number of $\operatorname{arcs}$ of $D$. Consider the following algorithm to find the maximum flow:
		
		\begin{algorithm}[H]
			\caption{Algorithm 1}
			Initialize $f \leftarrow 0, \Delta \leftarrow 2^{\lfloor \log U \rfloor}$ where $U$ denote the largest arc capacity, $\Delta$-residual network $D_f(\Delta)$ with residual capacities $u_f(a) \geq \Delta$ for all $a \in D_f(\Delta)$\;
			\While{$\Delta \geq 1$}{
				\While{$D_f(\Delta)$ contains an $(s-t)$-path}{
					Identify an $(s-t)$-path $P$ in $D_f(\Delta)$\;
					$f^\star \leftarrow \min_{a \in P} u_f(a)$\;
					Obtain new flow $f$ by augmenting $f^\star$ units of flow along $P$ and update $D_f(\Delta)$ and $u_f$\;
				}
				$\Delta \leftarrow \Delta/2$\;
			}
			\Return{flow $f$}
		\end{algorithm}
		
		\begin{enumerate}
			\item Prove that upon termination, the algorithm outputs the maximum flow.
			\item We refer to a phase of the algorithm during which $\Delta$ remains constant as a scaling phase. Show that the algorithm terminates after $O(\log U)$ scaling phases.
			\item Let $f^{\prime}$ be the value of the flow at the end of a scaling phase for some fixed $\Delta$. Let $D_{f^{\prime}}$ be the residual network of $D(V, A)$ with respect to flow $f^{\prime}$. Prove that the capacity of the minimum cut i n $D_{f^{\prime}}$ is at most $m \Delta$.
			\item Show that the algorithm performs at most $2 m$ augmentations per scaling phase. Hint: Build on (c).
			\item Conclude that the total running time of the algorithm is $O\left(m^{2} \log U\right)$.
		\end{enumerate}
	
	\textbf{Answer}
	
	
	\subsection{Part 1}
	Let $D=(V,A)$ be a directed network with integer capacities $u(a)\in \mathbb{Z}_{\ge 0}$.
	For a feasible flow $f$, let $u_f(a)$ denote residual capacities in the usual residual
	network $D_f$.
	For a parameter $\Delta\ge 1$, the {$\Delta$-residual network} $D_f(\Delta)$ is the
	subgraph of $D_f$ containing only residual arcs $a$ with $u_f(a)\ge \Delta$.
	
	Algorithm 1 (capacity scaling) maintains $\Delta$ and repeatedly augments along an
	$s$--$t$ path in $D_f(\Delta)$, then halves $\Delta$.
	
	Let $m:=|A|$ and let $U:=\max_{a\in A} u(a)$.
	
		The algorithm terminates only after reaching $\Delta<1$, i.e.\ after completing the
		$\Delta=1$ scaling phase. During the $\Delta=1$ phase, $D_f(1)$ contains {exactly}
		all residual arcs with positive residual capacity, since $u_f(a)\ge 1$ iff $u_f(a)>0$
		(because capacities and thus residual capacities remain integers throughout).
		
		The inner loop for $\Delta=1$ stops only when $D_f(1)$ contains no $s$--$t$ path, i.e.\
		when there is no augmenting path in the full residual network $D_f$.
		By the standard Ford--Fulkerson optimality criterion, a feasible flow is maximum iff
		its residual network contains no $s$--$t$ augmenting path. Hence the output flow is
		a maximum flow. 
		
	
	\subsection{Part 2}
		
		Initialization sets $\Delta = 2^{\lfloor \log_2 U\rfloor}$, i.e.\ the largest power of $2$
		not exceeding $U$. Each scaling phase replaces $\Delta$ by $\Delta/2$, so the sequence of
		$\Delta$ values is
		\[
		2^{\lfloor \log_2 U\rfloor},\;
		2^{\lfloor \log_2 U\rfloor-1},\;
		\dots,\;
		2^0 = 1.
		\]
		Thus the number of distinct $\Delta$ values (phases) is $\lfloor \log_2 U\rfloor+1$,
		which is $O(\log U)$. 
		
		
		\subsection{Part 3}
		
		Fix a scaling phase with parameter $\Delta$, and let $f'$ be the flow at the end of this
		phase, i.e.\ when the inner while-loop for this $\Delta$ has terminated.
		Then by definition, $D_{f'}(\Delta)$ has {no} $s$--$t$ path.
		
		Let $S$ be the set of vertices reachable from $s$ in $D_{f'}(\Delta)$, and let $T=V\setminus S$.
		Since $t$ is not reachable, we have $t\in T$, so $(S,T)$ is an $s$--$t$ cut.
		
		Consider any residual arc $a$ of $D_{f'}$ crossing from $S$ to $T$.
		If $u_{f'}(a)\ge \Delta$, then $a$ would be present in $D_{f'}(\Delta)$ and would allow
		reachability to enter $T$, contradicting the definition of $S$. Therefore every residual
		arc $a$ crossing $(S,T)$ satisfies
		\[
		u_{f'}(a) < \Delta.
		\]
		The capacity of the cut $(S,T)$ in the residual network $D_{f'}$ is
		\[
		c_{f'}(S,T) \;:=\; \sum_{a\in \delta^+_{D_{f'}}(S)} u_{f'}(a),
		\]
		where $\delta^+_{D_{f'}}(S)$ are residual arcs leaving $S$.
		Each summand is $<\Delta$, and the number of residual arcs is at most $2m$ (each original
		arc yields at most one forward and one backward residual arc). In particular,
		\[
		c_{f'}(S,T) \;\le\; (2m)\Delta.
		\]
		If one defines $m$ to be the number of {residual} arcs in $D_f$ (as some notes do),
		then the bound is exactly $m\Delta$; with $m=|A|$ it is $2m\Delta$.
		In either convention, this shows the minimum $s$--$t$ cut capacity in $D_{f'}$ is
		$O(m\Delta)$. 
		
		 \subsection{Part 4}
		
		Fix a scaling phase with parameter $\Delta$ and let $f'$ be the flow at its end.
		Let $f$ be the flow at the {start} of this $\Delta$-phase.
		Every augmentation in this phase sends $f^\star=\min_{a\in P} u_f(a)\ge \Delta$ units
		of flow, because the path $P$ lies in $D_f(\Delta)$ and all arcs on $P$ have residual
		capacity at least $\Delta$.
		
		Let $|f|$ denote the value of the flow.
		Consider the residual network at the end of the phase, $D_{f'}$.
		By the max-flow/min-cut theorem applied to the residual network (equivalently, by
		standard flow arguments),
		\[
		|f^*| - |f'| \;=\; \text{(max additional flow possible in }D_{f'}\text{)}
		\;\le\; \text{(min cut capacity in }D_{f'}\text{)}.
		\]
		By part (c), the minimum cut capacity in $D_{f'}$ is at most $O(m\Delta)$, hence
		\[
		|f^*|-|f'| \;\le\; 2m\Delta \qquad (\text{using } m=|A|).
		\]
		Also, throughout this phase, the total increase in flow value is
		\[
		|f'|-|f| \;\le\; |f^*|-|f| \;\le\; (|f^*|-|f'|) + (|f'|-|f|)
		\]
		but more directly we can bound the \emph{number} of augmentations $k$ in this phase by
		\[
		k\Delta \;\le\; |f'|-|f| \;\le\; |f^*|-|f| \;\le\; |f^*|-|f'| + (|f'|-|f|),
		\]
		which is not tight as written. The standard clean argument is:
		
		At the {end} of the phase, there is no $\Delta$-augmenting path, so any further
		augmentation must use a path with bottleneck $<\Delta$, i.e.\ the remaining augmentable
		flow in $D_{f'}$ is at most the min-cut capacity in $D_{f'}$, which by (c) is $\le 2m\Delta$.
		But each augmentation \emph{during the phase} increases $|f|$ by at least $\Delta$.
		Hence, if there were more than $2m$ augmentations in the phase, the flow value would
		have increased by more than $2m\Delta$, contradicting that once the phase ends, the
		residual network has an $s$--$t$ cut of capacity $\le 2m\Delta$ (so the total additional
		flow possible from that point is $\le 2m\Delta$).
		Therefore the number of augmentations in a phase is at most $2m$ (or $m$ under the
		alternative convention of counting residual arcs). 
		
	
	\subsection{Part 5}
		
		There are $O(\log U)$ scaling phases by part (b).
		By part (d), each phase performs $O(m)$ augmentations.
		
		If we find each augmenting path in $D_f(\Delta)$ by BFS/DFS in $O(m)$ time (since
		$|A|=m$ and $|V|\le m+1$ in a connected flow instance), and update residual capacities
		along the path in $O(m)$ worst-case but in fact $O(|P|)\le O(m)$, then each augmentation
		costs $O(m)$ time.
		
		Thus total time is
		\[
		O(\#\text{phases})\cdot O(\#\text{augs/phase})\cdot O(\text{time/aug})
		\;=\;
		O(\log U)\cdot O(m)\cdot O(m)
		\;=\;
		O(m^2\log U).
		\]


	
		\clearpage
		
		\section{Question 5}
		
		{\color{deepred}\fontspec{Georgia} Edmonds-Karp Algorithm}
		
		The Edmonds-Karp algorithm implements the Ford-Fulkerson algorithm by always choosing a shortest augmenting path in the residual network. Suppose instead that the Ford-Fulkerson algorithm chooses a widest augmenting path: an augmenting path with the greatest residual capacity. Assume that $G=(V, E)$ is a flow network with source $s$ and sink $t$, that all capacities are integer, and that the largest capacity is $C$. In this problem, you will show that choosing a widest augmenting path results in at most $|E| \ln \left|f^{*}\right|$ augmentations to find a maximum flow $f^{*}$.
		
		\begin{enumerate}
			\item Show how to adjust Dijkstra's algorithm to find the widest augmenting path in the residual network.
			\item Show that a maximum flow in $G$ can be formed by successive flow augmentations along at most $|E|$ paths from $s$ to $t$.
			\item Given a flow $f$, argue that the residual network has an augmenting path $p$ with residual capacity $(p) \geq\left(\left|f^{*}\right|-|f|\right) /|E|$.
			\item Assuming that each augmenting path is a widest augmenting path, let $f_{i}$ be the flow after augmenting the flow by the $i$ th augmenting path, where $f_{0}$ has $f(u, v)=0$ for all edges $(u, v)$. Show that $\left|f^{*}\right|-\left|f_{i}\right| \leq\left|f^{*}\right|(1-1 /|E|)^{i}$.
			\item Show that $\left|f^{*}\right|-\left|f_{i}\right|<\left|f^{*}\right| e^{-i /|E|}$.
			\item Conclude that after the flow is augmented at most $|E| \ln \left|f^{*}\right|$ times, the flow is a maximum flow.
		\end{enumerate}
	
	
		\textbf{Answer}
		
		\subsection{Part 1}

		
		Let $G=(V,E)$ be a flow network with integer capacities, source $s$, sink $t$.
		Given a flow $f$, let $G_f$ be the residual network and for an $s$--$t$ path $p$ in $G_f$
		let its residual capacity be
		\[
		c_f(p):=\min_{e\in p} c_f(e).
		\]
		A \emph{widest} augmenting path is an $s$--$t$ path maximizing $c_f(p)$.

			This is the standard \emph{maximum-bottleneck} (a.k.a.\ widest path) problem.
			Maintain for each vertex $v$ a label
			\[
			\mathrm{width}[v] = \max_{p:s\leadsto v} \min_{e\in p} c_f(e),
			\]
			the best bottleneck value achievable from $s$ to $v$.
			Initialize $\mathrm{width}[s]=+\infty$ and $\mathrm{width}[v]=0$ for $v\ne s$.
			Use a max-priority queue keyed by $\mathrm{width}[\cdot]$.
			
			Repeatedly extract a vertex $u$ of maximum $\mathrm{width}[u]$ not yet finalized, and
			relax each residual arc $(u,v)$ by
			\[
			\mathrm{cand} := \min\{\mathrm{width}[u],\, c_f(u,v)\},
			\qquad
			\text{if }\mathrm{cand}>\mathrm{width}[v]\text{ then update } \mathrm{width}[v]=\mathrm{cand}
			\text{ and set }\mathrm{parent}[v]=u.
			\]
			When $t$ is extracted (or the queue empties), $\mathrm{width}[t]$ equals the maximum
			$b$ such that there exists an $s$--$t$ path with bottleneck at least $b$, i.e.\ a widest
			augmenting path can be recovered by following $\mathrm{parent}[\cdot]$ pointers.
			
			Correctness follows from the same cut/greedy argument as Dijkstra:
			once a vertex $u$ with maximum tentative $\mathrm{width}[u]$ is extracted, no other
			path can improve its bottleneck value, because any alternative path to $u$ must pass
			through a vertex whose current label is $\le \mathrm{width}[u]$ and then take a final
			edge of residual capacity $\le$ that bottleneck. Hence $\mathrm{width}[u]$ is final. Here running time is $O(|E|\log |V|)$ with a binary heap.
			
			
		
		
		\subsection{Part 2}
		
		Let $f^*$ be any maximum flow. Consider the difference $g := f^*-f$ (in particular take
		$f\equiv 0$ if you like). One can decompose $g$ into flows on simple $s$--$t$ paths plus
		cycles. A constructive proof:
		
		Form a directed multigraph on $E$ where each original directed edge $e$ has value $g(e)$
		(possibly $0$). While there exists a vertex with positive excess other than $s$ or $t$,
		send flow along some outgoing edge; because total excess is $0$ and only $s$ has net
		positive supply and $t$ has net demand, repeated walk-following from $s$ must eventually
		reach $t$ (yielding an $s$--$t$ path) or repeat a vertex (yielding a directed cycle).
		In either case, push $\delta$ units along the found path/cycle, where $\delta$ is the
		minimum $g(e)$ on it, and subtract it from the involved edges. This eliminates at least
		one edge (an edge with $g(e)=\delta$ becomes $0$). Therefore, the process terminates after
		at most $|E|$ such path/cycle extractions.
		
		Discarding cycles (they do not affect the $s$--$t$ value), we obtain a representation
		\[
		f^* = \sum_{k=1}^{m} x_k \chi^{p_k} \;+\; \sum_{\ell} y_\ell \chi^{C_\ell},
		\qquad m \le |E|,
		\]
		where each $p_k$ is an $s$--$t$ path, $C_\ell$ a directed cycle, and $x_k,y_\ell>0$.
		Hence the \emph{value} $|f^*|$ can be achieved by augmenting along at most $|E|$ paths.
		
		
		\subsection{Part 3}
		
		Fix a current flow $f$ and an optimal flow $f^*$.
		Consider $g := f^* - f$ expressed on the original edges as above.
		In the residual network $G_f$, each unit of additional flow that takes $f$ to $f^*$
		corresponds to sending flow along either a forward residual arc $(u,v)$ (if $g(u,v)>0$)
		or a backward residual arc $(v,u)$ (if $g(u,v)<0$). Thus $g$ can be viewed as a feasible
		$s$--$t$ flow in $G_f$ of value
		\[
		|g| = |f^*|-|f|.
		\]
		By part (2), this residual $s$--$t$ flow $g$ can be decomposed into at most $|E|$
		$s$--$t$ paths (ignoring cycles), say with path-values $\alpha_1,\dots,\alpha_m$ where
		$m\le |E|$ and $\sum_{j=1}^m \alpha_j = |f^*|-|f|$.
		Therefore some path has value at least the average:
		\[
		\max_j \alpha_j \;\ge\; \frac{|f^*|-|f|}{m} \;\ge\; \frac{|f^*|-|f|}{|E|}.
		\]
		But the value of an $s$--$t$ path flow is at most its bottleneck, and in a path-flow
		construction one can send exactly the bottleneck. Hence there exists an augmenting path
		$p$ in $G_f$ with
		\[
		c_f(p) \;\ge\; \frac{|f^*|-|f|}{|E|}.
		\]
		
		
		\subsection{Part 4}
		
		Let $f_i$ be the flow after $i$ widest augmentations, with $f_0\equiv 0$.
		Let $\Delta_i := |f^*| - |f_i|$ be the remaining gap to optimality.
		By part (3), in the residual network of $f_i$ there exists \emph{some} augmenting path
		with residual capacity at least $\Delta_i/|E|$.
		Since we choose a \emph{widest} augmenting path, the chosen path has residual capacity
		at least that much. Thus the augmentation increases the flow value by
		\[
		|f_{i+1}| - |f_i| \;\ge\; \frac{\Delta_i}{|E|}.
		\]
		Equivalently,
		\[
		\Delta_{i+1}
		= |f^*|-|f_{i+1}|
		\le |f^| - \left(|f_i| + \frac{\Delta_i}{|E|}\right)
		= \Delta_i\left(1-\frac{1}{|E|}\right).
		\]
		Iterating from $\Delta_0 = |f^*|$ gives
		\[
		|f^*| - |f_i|
		= \Delta_i
		\le |f^*|\left(1-\frac{1}{|E|}\right)^i.
		\]
		
		
		\subsection{Part 5}
		
		Use the standard inequality $1-x \le e^{-x}$ for all real $x$.
		With $x=1/|E|$ we obtain
		\[
		\left(1-\frac{1}{|E|}\right)^i \le e^{-i/|E|},
		\]
		hence
		\[
		|f^*|-|f_i|
		\le |f^*|\left(1-\frac{1}{|E|}\right)^i
		< |f^*| e^{-i/|E|}.
		\]
		
		
		\subsection{Part 6}
		
		Capacities are integers, and Ford--Fulkerson augmentations along residual paths preserve
		integrality of the flow. Therefore $|f_i|$ and $|f^*|$ are integers, so the gap
		$\Delta_i = |f^*|-|f_i|$ is a nonnegative integer.
		
		If
		\[
		i \;\ge\; |E|\ln |f^*|,
		\]
		then
		\[
		|f^*|-|f_i| \;<\; |f^*| e^{-i/|E|}
		\;\le\; |f^*| e^{-\ln|f^*|} \;=\; 1.
		\]
		Since $\Delta_i$ is an integer and $\Delta_i<1$, we must have $\Delta_i=0$, i.e.
		$|f_i|=|f^*|$ and $f_i$ is a maximum flow.
		
		Thus widest-augmenting-path Ford--Fulkerson terminates after at most
		\[
		|E|\ln|f^*|
		\]
		augmentations. 
	
	\clearpage
	
	\section{Question 6}
	
	{\color{deepred}\fontspec{Georgia} Path Cover}
	
	A path cover of a directed graph $G=(V, E)$ is a set $P$ of vertex-disjoint paths such that every vertex in $V$ is included in exactly one path in $P$. Paths may start and end anywhere, and they may be of any length, including 0. A minimum path cover of $G$ is a path cover containing the fewest possible paths.
	
	\begin{itemize}
		\item Give an efficient algorithm to find a minimum path cover of a directed acyclic graph $G=(V, E)$. (Assuming that $V=1,2, \ldots, n$, construct a flow network based on the graph $G^{\prime}=\left(V^{\prime}, E^{\prime}\right)$, where
	\end{itemize}
	
	$$
	\begin{aligned}
		V^{\prime} & =x_{0}, x_{1}, \ldots, x_{n} \cup y_{0}, y_{1}, \ldots, y_{n} \\
		E^{\prime} & =\left(x_{0}, x_{i}\right) i \in V \cup\left(y_{i}, y_{0}\right) i \in V \cup\left(x_{i}, y_{j}\right)(i, j) \in E
	\end{aligned}
	$$
	
	and run a maximum-flow algorithm.)
	
	\begin{itemize}
		\item Does your algorithm work for directed graphs that contain cycles? Explain.
	\end{itemize}

\textbf{Answer}

\subsection{Part 1}


A \emph{path cover} of a directed graph $G=(V,E)$ is a set of vertex-disjoint directed paths
covering all vertices. In a DAG, the minimum path cover can be found by reducing to a
maximum matching problem in a bipartite graph, which can be solved by max flow.

\medskip
Assume $V=\{1,2,\dots,n\}$. Build the directed network $G'=(V',E')$ exactly as suggested:
\[
V'=\{x_0,x_1,\dots,x_n\}\ \cup\ \{y_0,y_1,\dots,y_n\},
\]
\[
E'=\{(x_0,x_i): i\in V\}\ \cup\ \{(y_i,y_0): i\in V\}\ \cup\ \{(x_i,y_j): (i,j)\in E\}.
\]
Give every arc capacity $1$.

Let $s:=x_0$ and $t:=y_0$. Compute a maximum $s$-$t$ flow of value $F$.

\medskip
\emph{Claim 1 (flow $\leftrightarrow$ matching).}
The arcs $(x_i,y_j)$ carrying one unit of flow form a matching $M$ in the bipartite graph
with left part $\{x_1,\dots,x_n\}$ and right part $\{y_1,\dots,y_n\}$ and edges
$x_i y_j$ whenever $(i,j)\in E$.

\emph{Proof.}
Capacity $1$ and flow conservation imply each $x_i$ can send at most one unit to the
right side (since the only incoming arc to $x_i$ is $(s,x_i)$ of capacity $1$),
and each $y_j$ can receive at most one unit from the left side (since the only outgoing
arc from $y_j$ is $(y_j,t)$ of capacity $1$). Thus the used $(x_i,y_j)$ edges are
vertex-disjoint on both sides, i.e.\ a matching. Conversely, any matching yields a feasible
flow of the same value by sending $1$ along $s\to x_i \to y_j \to t$ for each matched pair.

\medskip
\emph{Claim 2 (minimum path cover size).}
If $G$ is a DAG, then
\[
\text{size of a minimum path cover} \;=\; n - |M|,
\]
where $M$ is a maximum matching (equivalently $|M|=F$).

\emph{Proof.}
Every path cover $P$ induces a matching as follows:
for each path $v_1\to v_2\to \cdots \to v_k$ in $P$, include the pairs
$(v_1,v_2),(v_2,v_3),\dots,(v_{k-1},v_k)$ in the matching.
Because paths are vertex-disjoint, each vertex has at most one successor and at most
one predecessor inside the cover, so these pairs are disjoint on both sides: a matching.
If the cover has $|P|$ paths, then the total number of such pairs is
\[
(n-|P|),
\]
since each of the $|P|$ paths of length $k$ contributes $(k-1)$ pairs and
$\sum (k)=n$.

Hence $|M|\ge n-|P|$ for every path cover $P$, so $|P|\ge n-|M|$.

Conversely, given a matching $M$, create a directed graph $H$ on vertex set $V$ by adding
exactly the matched edges $(i,j)\in M$ as directed edges $i\to j$.
Because $M$ matches each vertex at most once as a predecessor and at most once as a
successor, every vertex in $H$ has outdegree $\le 1$ and indegree $\le 1$.
When $G$ is acyclic, $H$ is also acyclic, so every component of $H$ is a directed path.
These paths are vertex-disjoint and cover all vertices, giving a path cover with exactly
\[
n-|M|
\]
paths (each matched edge reduces the number of paths by $1$). Thus a minimum path cover
has size $n-|M|$.

\medskip
\noindent
\fbox{
	\begin{minipage}{0.97\linewidth}
		\textbf{Algorithm}
		\begin{enumerate}
			\item Build $G'$ as above and set all capacities to $1$.
			\item Run any polynomial-time max-flow algorithm to get a maximum flow value $F$.
			\item Let $M:=\{(i,j)\in E : f(x_i,y_j)=1\}$.
			\item Output a path cover by linking each matched edge $i\to j$.
			Concretely, start a path at every vertex with no matched incoming edge,
			and repeatedly follow the unique matched outgoing edge (if any) until it stops.
		\end{enumerate}
		The number of paths returned is $n-F$, which is minimum by the claims above.
	\end{minipage}
}


\medskip
The network has $2n+2$ vertices and $2n+|E|$ edges.
With unit capacities, one may use Dinic in $O(\sqrt{|V'|}\,|E'|)$ for bipartite matching
(Hopcroft--Karp viewpoint), i.e.\ $O(\sqrt{n}\,(|E|+n))$, or any standard max-flow bound.

\bigskip


\subsection{Part 2}

The {reduction to matching via $G'$ still computes a maximum matching} in the bipartite
graph (Claim 1), but the equality
\[
\text{min path cover size} = n-|M|
\]
{need not hold} when $G$ has cycles.

Reason: from a matching $M$ we form $H$ with indegree/outdegree $\le 1$, but if $G$ has
cycles, $H$ may contain a directed cycle. A directed cycle is \emph{not} a directed path,
so it cannot be part of a path cover. Thus the step ``matching $\Rightarrow$ path cover
with $n-|M|$ paths'' can fail.

In fact, for general directed graphs, finding a minimum path cover (with vertex-disjoint
directed paths covering all vertices) is NP-hard; the clean matching formula and the
flow-based algorithm rely essentially on acyclicity (so that $H$ decomposes into paths
rather than paths-plus-cycles).

		
		
	\end{document}
